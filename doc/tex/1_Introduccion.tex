\capitulo{1}{Introducción}

Gracias al avance de las técnicas y algoritmos de minería de datos, disciplinas no directamente relacionadas con la computación se han ido beneficiando de las ciencias de datos, a modo particular, la medicina está desarrollándose hacia modelos más preventivos gracias a las predicciones que se pueden generar utilizando estos métodos.

Es por este motivo que podemos ver en la literatura científica de los últimos años como se relaciona medicina y ciencia de datos, por ejemplo en estudios de detección de caídas~\cite{tolkiehn2011fall} o la motorización del sueño para la prevención de apneas~\cite{kortelainen2012sleepmonitoring}. En este trabajo de fin de grado, el objetivo es la detección de crisis epilépticas adscrito al proyecto homónimo vencedor del concurso universitario \textit{Desafío Universidad Empresa}~\cite{radio:radio_amiga_burgos_2018}

Aunque el análisis y detección de crisis epilépticas de manera automática ha sido ampliamente explorada por la comunidad científica esta se ha centrado o en el uso de pulseras inteligentes~\cite{ramgopal2014product_review} o el uso de encefalogramas~(\textit{EEG})~\cite{jeppesen2017modified,kumar2014epilepticeeg,tzallas2012review} para la predicción de estos eventos. Por tanto, ante pacientes con diversos problemas de salud que impiden el uso de pulseras y de dispositivos de control de actividad cerebral se realiza este proyecto que enfoca la detección de las crisis epilépticas en el uso de sensores de presión y biométricos en la cama donde duerme el paciente.

\section{Material adjunto}
Junto a esta memoria se incluyen:

\begin{itemize}
	\item \textbf{Cuaderno de investigación} con la evolución y pasos realizados durante la investigación junto con los resultados de cada experimento.
	\item \textbf{Anexos} donde se incluyen:
		\begin{itemize}
			\item Plan de proyectos
			\item Requisitos del sistema
			\item Diseño del sistema
			\item Manual para el programador
			\item Manual para el usuario
		\end{itemize}
	\item \textbf{API REST} en \textbf{Python-Flask}
	\item \textbf{Experimentos} en \textit{Jupyter Notebook}
\end{itemize}

Además se puede acceder a través de internet a la página web en producción~\cite{garrido2019web} y al repositorio GitHub del proyecto~\cite{garrido2019repo}.