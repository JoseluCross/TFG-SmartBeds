\apendice{Plan de Proyecto Software}

\section{Introducción}

\section{Planificación temporal}

El proyecto se desarrolló siguiendo una metodología ágil, ligeramente basada en \textit{Scrum}. Se dividió el progreso en \textit{Sprints}, cada uno con una serie de tareas y su estimación en esfuerzo. Esta estimación o peso, se ha evaluado según la dificultad que se preveía tener, no como una medida horaria, esto es debido que algunas tareas simples tardan más en desarrollarse por la necesidad de la ejecución automática mientras que otras que requirieron mucho más trabajo se ejecutaron mucho antes.

A continuación se mostrarán las listas de las tareas realizadas con los enlaces a las \textit{issues} del repositorio \textit{GitHub}. Los \textit{Sprints} fueron:
\subsection{\textit{Sprint} 1 - 10/11/18 hasta 20/11/18}
En este primer \textit{Sprint} solamente hubo dos tareas:
\begin{enumerate}
	\item \href{https://github.com/joselucross/TFG-SmartBeds/issues/1}{Hacer exploración bibliográfica}
	\item \href{https://github.com/joselucross/TFG-SmartBeds/issues/2}{Configurar repositorio de Git}
\end{enumerate}
\subsection{\textit{Sprint} 2 - 21/11/18 hasta 05/12/18}
Se desarrollaron 4 tareas, todas de investigación, por lo que no hay \textit{commits} asociados, para esto las tareas fueron:

\begin{enumerate}\addtocounter{enumi}{2}
	\item \href{https://github.com/joselucross/TFG-SmartBeds/issues/3}{Lectura de "Automated Epileptic Seizure Detection Methods: A Review Study"}
	\item \href{https://github.com/joselucross/TFG-SmartBeds/issues/4}{Exploración Bibliográfica - Orientado a caídas}
	\item \href{https://github.com/joselucross/TFG-SmartBeds/issues/5}{ Lectura del primer tema de "Minería de Datos"}
	\item \href{https://github.com/joselucross/TFG-SmartBeds/issues/6}{ Configurar VPN en Archlinux}
\end{enumerate}

\subsection{\textit{Sprint} 3 - 06/12/18 hasta 21/12/18}
En este \textit{sprint} se comenzó la documentación migrando las plantillas al repositorio y continuó la investigación a un área más técnica, las tareas fueron:

\begin{enumerate}\addtocounter{enumi}{6}
	\item \href{https://github.com/joselucross/TFG-SmartBeds/issues/7}{Iniciar documentación}
	\item \href{https://github.com/joselucross/TFG-SmartBeds/issues/8}{Tabla de extracción de características}
	\item \href{https://github.com/joselucross/TFG-SmartBeds/issues/9}{ Búsqueda de librerías con funciones sofisticadas}
	
\end{enumerate}

\subsection{\textit{Sprint} 4 - 22/12/18 hasta 08/01/2019}
Este \textit{sprint} fue del aprendizaje de técnicas de minería de datos aplicada. Las tareas realizadas fueron:

\begin{enumerate}\addtocounter{enumi}{9}
	\item \href{https://github.com/joselucross/TFG-SmartBeds/issues/10}{ Insalacion de entorno python}
	\item \href{https://github.com/joselucross/TFG-SmartBeds/issues/11}{ Graficar datos mediante PCA}
	\item \href{https://github.com/joselucross/TFG-SmartBeds/issues/12}{ Aprender el uso de librerías}
\end{enumerate}

\subsection{\textit{Sprint} 5 - 09/01/2019 hasta 17/01/2019}
En este \textit{sprint} el objetivo fue probar distintas lineas de investigación para ver las características intrínsecas a los datos.

\begin{enumerate}\addtocounter{enumi}{12}
	\item \href{https://github.com/joselucross/TFG-SmartBeds/issues/13}{ Configurar acceso a gamma}
	\item \href{https://github.com/joselucross/TFG-SmartBeds/issues/14}{ Leer apuntes de Minería de Datos}
	\item \href{https://github.com/joselucross/TFG-SmartBeds/issues/15}{ Filtrado y suavizado de datos}
	\item \href{https://github.com/joselucross/TFG-SmartBeds/issues/16}{ Probar otras formas de proyección}
\end{enumerate}	

\subsection{\textit{Sprint} 6 - 18/01/2019 hasta 24/01/2019}
Tras descartar algunas proyecciones se exploraron las más prometedores y se probaron nuevos filtros y detección de anomalías. Las tareas fueron por consiguiente:

\begin{enumerate}\addtocounter{enumi}{16}
	\item \href{https://github.com/joselucross/TFG-SmartBeds/issues/17}{ Mejor preprocesamiento}
	\item \href{https://github.com/joselucross/TFG-SmartBeds/issues/18}{ Dibujado alrededor de las crisis}
	\item \href{https://github.com/joselucross/TFG-SmartBeds/issues/19}{ Probar formas de filtrado}
	\item \href{https://github.com/joselucross/TFG-SmartBeds/issues/20}{ Estudiar puntos clave de las proyecciones}
	\item \href{https://github.com/joselucross/TFG-SmartBeds/issues/21}{ Probar detección de anomalías por one-class}
\end{enumerate}

\subsection{\textit{Sprint} 7 - 25/01/2019 hasta 07/02/2019}
Este \textit{sprint} tuvo una duración mayor debido a que durante el periodo señalado se realizó un curso en la universidad donde se estudiaron las series temporales. Por este motivo el servidor donde se han estado realizando las ejecuciones no estaba disponible retrasando la ejecución de los experimentos.\\

Las tareas se centraron en comprobar las proyecciones más interesantes con datos estadísticos además de profundizar en \textit{One-Class} o detección de anomalías. Las tareas realizadas fueron:
\begin{enumerate}\addtocounter{enumi}{21}
	\item \href{https://github.com/joselucross/TFG-SmartBeds/issues/22}{Proyección LTSA con valores estadísticos}
	\item 
	\href{https://github.com/joselucross/TFG-SmartBeds/issues/23}{Documentar Sprints del 1 al 6}
	\item
	\href{https://github.com/joselucross/TFG-SmartBeds/issues/24}{Documentar investigación con Alicia}
	\item
	\href{https://github.com/joselucross/TFG-SmartBeds/issues/25}{One Class - Mejorar la investigación de este apartado}
	\item
	\href{https://github.com/joselucross/TFG-SmartBeds/issues/26}{Transformers}
	\item
	\href{https://github.com/joselucross/TFG-SmartBeds/issues/27}{Trasladar códigos a transformers}
\end{enumerate}
\section{Estudio de viabilidad}

\subsection{Viabilidad económica}

\subsection{Viabilidad legal}


