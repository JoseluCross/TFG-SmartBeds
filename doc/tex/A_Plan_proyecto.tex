\apendice{Plan de Proyecto Software}

\section{Introducción}

En este apéndice se expondrán los distintos \textit{sprints} que se han realizado y un estudio de la viabilidad del proyecto.

\section{Planificación temporal}

El proyecto se desarrolló siguiendo una metodología ágil basada en \textit{Scrum}. Se dividió el progreso en \textit{Sprints}, cada uno con una serie de tareas y su estimación en esfuerzo. Esta estimación o peso, se ha evaluado según la dificultad que se preveía tener, no como una medida horaria, esto se debe a que algunas tareas han sido sencillas para el desarrollador pero han requerido mayor tiempo de cómputo (e.g. Configurar experimentos), mientras otras tareas de ejecución mucho más rápida han requerido una mayor dedicación por el programador (e.g. Desarrollo de los test). Algunas tareas además se consideran épicas, estas son resúmenes de un proceso completo que abarca varias tareas y duran generalmente más de un \textit{sprint}. 

A continuación se mostrarán las listas de las tareas realizadas con los enlaces a las \textit{issues} del repositorio \textit{\href{https://github.com/jlgarridol/TFG-SmartBeds}{GitHub}}. Los \textit{Sprints} fueron:
\subsection{\textit{Sprint} 1 - 10/11/18 hasta 20/11/18}
En este primer \textit{Sprint} solamente hubo dos tareas:
\begin{enumerate}
	\item \href{https://github.com/joselucross/TFG-SmartBeds/issues/1}{Hacer exploración bibliográfica}
	\item \href{https://github.com/joselucross/TFG-SmartBeds/issues/2}{Configurar repositorio de Git}
\end{enumerate}
\subsection{\textit{Sprint} 2 - 21/11/18 hasta 05/12/18}
Se desarrollaron cuatro tareas, todas de investigación, por lo que no hay \textit{commits} asociados; las tareas fueron:

\begin{enumerate}\addtocounter{enumi}{2}
	\item \href{https://github.com/joselucross/TFG-SmartBeds/issues/3}{Lectura de <<Automated Epileptic Seizure Detection Methods: A Review Study>>}
	\item \href{https://github.com/joselucross/TFG-SmartBeds/issues/4}{Exploración Bibliográfica - Orientado a caídas}
	\item \href{https://github.com/joselucross/TFG-SmartBeds/issues/5}{ Lectura del primer tema de "Minería de Datos"}
	\item \href{https://github.com/joselucross/TFG-SmartBeds/issues/6}{ Configurar VPN en Archlinux}
\end{enumerate}

\subsection{\textit{Sprint} 3 - 06/12/18 hasta 21/12/18}
En este \textit{sprint} se comenzó la documentación migrando las plantillas al repositorio y continuó la investigación a un área más técnica; las tareas fueron:

\begin{enumerate}\addtocounter{enumi}{6}
	\item \href{https://github.com/joselucross/TFG-SmartBeds/issues/7}{Iniciar documentación}
	\item \href{https://github.com/joselucross/TFG-SmartBeds/issues/8}{Tabla de extracción de características}
	\item \href{https://github.com/joselucross/TFG-SmartBeds/issues/9}{ Búsqueda de librerías con funciones para el procesamiento de series temporales}
	
\end{enumerate}

\subsection{\textit{Sprint} 4 - 22/12/18 hasta 08/01/2019}
Este \textit{sprint} fue del aprendizaje de técnicas de minería de datos aplicada. Las tareas realizadas fueron:

\begin{enumerate}\addtocounter{enumi}{9}
	\item \href{https://github.com/joselucross/TFG-SmartBeds/issues/10}{ Instalación de entorno Python}
	\item \href{https://github.com/joselucross/TFG-SmartBeds/issues/11}{ Graficar datos mediante PCA}
	\item \href{https://github.com/joselucross/TFG-SmartBeds/issues/12}{ Aprender el uso de librerías \textit{Python} para la minería de datos}
\end{enumerate}

\subsection{\textit{Sprint} 5 - 09/01/2019 hasta 17/01/2019}
En este \textit{sprint} el objetivo fue probar distintas lineas de investigación para ver las características intrínsecas a los datos.

\begin{enumerate}\addtocounter{enumi}{12}
	\item \href{https://github.com/joselucross/TFG-SmartBeds/issues/13}{ Configurar acceso al computador del grupo ADMIRABLE \textit{Gamma}}
	\item \href{https://github.com/joselucross/TFG-SmartBeds/issues/14}{ Leer apuntes de Minería de Datos}
	\item \href{https://github.com/joselucross/TFG-SmartBeds/issues/15}{ Filtrado y suavizado de datos}
	\item \href{https://github.com/joselucross/TFG-SmartBeds/issues/16}{ Probar otras formas de proyección}
\end{enumerate}	

\subsection{\textit{Sprint} 6 - 18/01/2019 hasta 24/01/2019}
Tras descartar algunas proyecciones se exploraron las más prometedores y se probaron nuevos filtros y detección de anomalías. Las tareas fueron por consiguiente:

\begin{enumerate}\addtocounter{enumi}{16}
	\item \href{https://github.com/joselucross/TFG-SmartBeds/issues/17}{ Determinar el mejor preprocesamiento de los datos}
	\item \href{https://github.com/joselucross/TFG-SmartBeds/issues/18}{ Dibujado alrededor de las crisis}
	\item \href{https://github.com/joselucross/TFG-SmartBeds/issues/19}{ Probar formas de filtrado}
	\item \href{https://github.com/joselucross/TFG-SmartBeds/issues/20}{ Estudiar puntos clave de las proyecciones}
	\item \href{https://github.com/joselucross/TFG-SmartBeds/issues/21}{ Probar detección de anomalías por one-class}
\end{enumerate}

\subsection{\textit{Sprint} 7 - 25/01/2019 hasta 07/02/2019}
Este \textit{sprint} tuvo una duración mayor debido a que durante el periodo señalado se realizó un curso en la universidad donde se estudiaron las series temporales. Por este motivo, el servidor donde se han estado realizando las ejecuciones no estaba disponible retrasando la ejecución de los experimentos.

Las tareas se centraron en comprobar las proyecciones más interesantes con datos estadísticos además de profundizar en \textit{One-Class} o detección de anomalías. Las tareas realizadas fueron:
\begin{enumerate}\addtocounter{enumi}{21}
	\item \href{https://github.com/joselucross/TFG-SmartBeds/issues/22}{Proyección LTSA con valores estadísticos}~\cite{zhang2004principal}
	\item 
	\href{https://github.com/joselucross/TFG-SmartBeds/issues/23}{Documentar Sprints del 1 al 6}
	\item
	\href{https://github.com/joselucross/TFG-SmartBeds/issues/24}{Documentar investigación con Alicia Olivares}
	\item
	\href{https://github.com/joselucross/TFG-SmartBeds/issues/25}{One Class - Mejorar la investigación de este apartado}
	\item
	\href{https://github.com/joselucross/TFG-SmartBeds/issues/26}{Creación de transformadores de datos de \textit{scikit-learn} para el procesamiento de los datos}
	\item
	\href{https://github.com/joselucross/TFG-SmartBeds/issues/27}{Trasladar códigos existentes del procesado de datos a los transformadores de \textit{scikit-learn} creados}
\end{enumerate}

\subsection{\textit{Sprint} 8 - 8/02/2019 hasta 14/2/2019}
Este \textit{sprint} fue dedicado a seguir desarrollando el estudio de \textit{One Class} cuyos resultados se pueden ver en el apéndice \textit{Cuaderno de Investigación}. Las tareas fueron:

\begin{enumerate}\addtocounter{enumi}{27}
	\item \href{https://github.com/joselucross/TFG-SmartBeds/issues/28}{Particionar los datos}
	\item \href{https://github.com/joselucross/TFG-SmartBeds/issues/29}{Estudiar el funcionamiento de los algoritmos de \textit{one-class}\footnote{Tarea épica, siendo estas las tareas que engloban varias tareas más simples}}
	\item \href{https://github.com/joselucross/TFG-SmartBeds/issues/30}{Clasificador \textit{One-Class} entrenando con datos de crisis}
	\item \href{https://github.com/joselucross/TFG-SmartBeds/issues/31}{Clasificador \textit{One-Class} entrenando con datos de no crisis}
	\item \href{https://github.com/joselucross/TFG-SmartBeds/issues/32}{Preprocesado básico para \textit{One Class}}
	\begin{enumerate}
		\item Bruto
		\item Filtrado \textit{Butter} de 3 y 0.05
		\item Filtrado \textit{SavGol} de tamaño 15
		\item Concatenación de estadísticos en ventana 25 sobre bruto
		\item Concatenación de estadísticos con ventana 25 sobre \textit{Butter} de 3 y 0.05
		\item Concatenación de estadísticos con ventana 25 sobre \textit{SavGol} de tamaño 15
	\end{enumerate}
	\item \href{https://github.com/joselucross/TFG-SmartBeds/issues/33}{Testear clasificadores con otras crisis}
\end{enumerate}
Las tareas de la 30 a la 33 se incluyeron en la tarea épica 29 aunque esta tuvo tareas del siguiente \textit{sprint}

\subsection{\textit{Sprint} 9 - 15/2/2019 a 21/2/2019}
Este \textit{sprint} trató de completar la linea de investigación de \textit{One Class} abierta en la tarea épica 29 y documentar los resultados de \textit{sprints} anteriores. 

\begin{enumerate}\addtocounter{enumi}{33}
	\item
	\href{https://github.com/joselucross/TFG-SmartBeds/issues/34}{Documentar PCA y Proyecciones de 2-Variedad}
	\item
	\href{https://github.com/joselucross/TFG-SmartBeds/issues/35}{Entrenamiento \textit{One Class} con dos crisis}
	\item
	\href{https://github.com/joselucross/TFG-SmartBeds/issues/36}{Testeo \textit{One Class} con tercera crisis}
	\item
	\href{https://github.com/joselucross/TFG-SmartBeds/issues/37}{Calcular área bajo la curva con entrenamiento de una clase}
	\item
	\href{https://github.com/joselucross/TFG-SmartBeds/issues/38}{Visualización de constantes vitales}
\end{enumerate}
Son las tareas de la 35 a la 37 las que forman parte de la tarea épica 29.
\subsection{\textit{Sprint} 10 - 22/3/2019 a 28/3/2019}
Este \textit{sprint} estuvo centrado en la lectura y aprendizaje de nuevos métodos

\begin{enumerate}\addtocounter{enumi}{38}
	\item
	\href{https://github.com/joselucross/TFG-SmartBeds/issues/39}{Investigar sobre conjuntos de datos desequilibrados}
	\item
	\href{https://github.com/joselucross/TFG-SmartBeds/issues/40}{Aprender Weka}
	\item
	\href{https://github.com/joselucross/TFG-SmartBeds/issues/41}{Documentar \textit{One Class}}
\end{enumerate}
\subsection{\textit{Sprint} 11 - 01/03/2019 a 21/03/2019}
Este \textit{sprint} tuvo una duración mayor debido a la diversidad de experimentos y la presencia de varios días no lectivos. Se centró en el lanzamiento de experimentos con \textit{ensembles}.
\begin{enumerate}\addtocounter{enumi}{41}
	\item
	\href{https://github.com/joselucross/TFG-SmartBeds/issues/42}{Creación de test de transformers}
	\item
	\href{https://github.com/joselucross/TFG-SmartBeds/issues/43}{Filtrado SMOTE}
	\item
	\href{https://github.com/joselucross/TFG-SmartBeds/issues/43}{Pruebas con ensembles ya implementados}
	\item
	\href{https://github.com/joselucross/TFG-SmartBeds/issues/45}{Prueba de ensembles para desequilibrados}
\end{enumerate}

\subsection{\textit{Sprint} 12 - 22/03/2019 - 28/03/2019}
En esta ocasión comenzamos a reducir el esfuerzo sobre la investigación sobre un incremento en el diseño de la aplicación.

\begin{enumerate}\addtocounter{enumi}{45}
	\item
	\href{https://github.com/joselucross/TFG-SmartBeds/issues/46}{Exploración de herramientas}
	\item
	\href{https://github.com/joselucross/TFG-SmartBeds/issues/47}{Documentar resultados anteriores}
	\item
	\href{https://github.com/joselucross/TFG-SmartBeds/issues/48}{Diseño del servidor}
	\item
	\href{https://github.com/joselucross/TFG-SmartBeds/issues/49}{Ejecutar experimentos Weka restantes}
\end{enumerate}

\subsection{\textit{Sprint} 13 - 29/03/2019 - 08/04/2019}
Este \textit{sprint} se centró en la definición de requisitos y de realizar experimentos con los resultados de Alicia Olivares. Sin embargo, estos experimentos no pudieron ser realizados al descubrir un fallo en la investigación previa, por lo que no se realizaron quedando pendientes ante nuevos datos. Por tanto, las tareas que realmente fueron realizadas fueron:

\begin{enumerate}\addtocounter{enumi}{49}
	\item
	\href{https://github.com/joselucross/TFG-SmartBeds/issues/50}{Crear la simulación de una cama para la obtención en tiempo real de datos}
	\addtocounter{enumi}{1}
	\item
	\href{https://github.com/joselucross/TFG-SmartBeds/issues/52}{Diseño de casos de uso}
	\item
	\href{https://github.com/joselucross/TFG-SmartBeds/issues/53}{Diseño de los datos para la API}
	\item
	\href{https://github.com/joselucross/TFG-SmartBeds/issues/54}{Diseño de la base de datos}
	\addtocounter{enumi}{2}
	\item
	\href{https://github.com/joselucross/TFG-SmartBeds/issues/56}{Diseño de la API}
	\item
	\href{https://github.com/joselucross/TFG-SmartBeds/issues/57}{Requisitos funcionales}
\end{enumerate} 

\subsection{\textit{Spint} 14 - 09/04/2019 - 11/04/2019 }
Este \textit{sprint} se concluyeron la creación de los experimentos que quedaron pendientes en el \textit{sprint} anterior como seguir desarrollando componentes de la aplicación. 

Sin embargo, muchas partes de este \textit{sprint} fueron quedando en \textit{icebox} tras algunas complicaciones en los experimentos y la nueva linea de investigación con \textit{PRC} que se puede ver en el \textit{Cuaderno de investigación} adjuntado. 

\begin{enumerate}\addtocounter{enumi}{50}
	\item
	\href{https://github.com/jlgarridol/TFG-SmartBeds/issues/51}{Experimentos Weka con los datos de series temporales}
	\addtocounter{enumi}{3}
	\item
	\href{https://github.com/jlgarridol/TFG-SmartBeds/issues/55}{Implementación de la base de datos}
	\addtocounter{enumi}{2}
	\item
	\href{https://github.com/jlgarridol/TFG-SmartBeds/issues/58}{Creación de experimentos}
	\item
	\href{https://github.com/jlgarridol/TFG-SmartBeds/issues/59}{Lanzamiento de experimentos}
	\item
	\href{https://github.com/jlgarridol/TFG-SmartBeds/issues/60}{Documentación de resultados}
	\item
	\href{https://github.com/jlgarridol/TFG-SmartBeds/issues/61}{Experimentos con conjuntos de datos desequilibrados\footnote{Tarea épica}}
	\item
	\href{https://github.com/jlgarridol/TFG-SmartBeds/issues/62}{Ultimar detalles del cuaderno de trabajo}
	\item
	\href{https://github.com/jlgarridol/TFG-SmartBeds/issues/63}{Arreglar documentación de casos de uso}
	\item
	\href{https://github.com/jlgarridol/TFG-SmartBeds/issues/64}{Completar diseño de bases de datos}
	\item
	\href{https://github.com/jlgarridol/TFG-SmartBeds/issues/65}{Trabajar en formato de latex}
	\item 
	\href{https://github.com/jlgarridol/TFG-SmartBeds/issues/66}{Crear interfaz de la api}
	\item 
	\href{https://github.com/jlgarridol/TFG-SmartBeds/issues/67}{Creación y documentación de prototipos de interfaz}
\end{enumerate}

\subsection{\textit{Spint} 15 - 12/04/2019 - 2/05/2019 }
En este \textit{sprint} se comenzó la implementación de la API así como la realización de test automáticos de la misma. También se comenzó a preparar el servidor para que fuese accesible desde Internet.

\begin{enumerate}\addtocounter{enumi}{67}
	\item 
	\href{https://github.com/jlgarridol/TFG-SmartBeds/issues/68}{Programación de la API para la gestión y distribución de datos de la aplicación}
	\item 
	\href{https://github.com/jlgarridol/TFG-SmartBeds/issues/69}{Test sobre la API}
	\item 
	\href{https://github.com/jlgarridol/TFG-SmartBeds/issues/70}{Acceso web a la API}
	\item 
	\href{https://github.com/jlgarridol/TFG-SmartBeds/issues/71}{Documentar Sprints anteriores}
	\item 
	\href{https://github.com/jlgarridol/TFG-SmartBeds/issues/72}{Configurar servidor}
	\item 
	\href{https://github.com/jlgarridol/TFG-SmartBeds/issues/73}{Entorno Flask}
\end{enumerate}

\subsection{\textit{Spint} 16 - 3/05/2019 - 09/05/2019}
Este \textit{sprint} consistió en continuar el desarrollo del servidor.

\begin{enumerate}\addtocounter{enumi}{73}
	\item 
	\href{https://github.com/jlgarridol/TFG-SmartBeds/issues/74}{Implementar hilo de procesamiento}
	\item 
	\href{https://github.com/jlgarridol/TFG-SmartBeds/issues/75}{Distribución de paquetes por SocketIO}
	\item 
	\href{https://github.com/jlgarridol/TFG-SmartBeds/issues/76}{Uso del clasificador de Alicia Olivares}
	\item 
	\href{https://github.com/jlgarridol/TFG-SmartBeds/issues/77}{Actualización de Chart.js es muy lenta}~[BUG]
	\item 
	\href{https://github.com/jlgarridol/TFG-SmartBeds/issues/78}{Lanzar proyecto en el servidor}
\end{enumerate}

\subsection{\textit{Spint} 17 - 10/05/2019 - 16/05/2019}
Este \textit{sprint} se centró en el desarrollo de la vista así como reparar algunos \textit{bugs} que fueron apareciendo. También se intento utilizar \textit{Heroku} pero al conseguir eliminar las limitaciones del servidor se desechó la idea.

\begin{enumerate}\addtocounter{enumi}{79}
	\item 
	\href{https://github.com/jlgarridol/TFG-SmartBeds/issues/80}{Crear conexiones SQL bajo demanda por \texttt{request}}
	\item 
	\href{https://github.com/jlgarridol/TFG-SmartBeds/issues/81}{Hacer todas las ventanas}
	\item 
	\href{https://github.com/jlgarridol/TFG-SmartBeds/issues/82}{Crear cama simulada de los datos alrededor de las crisis}
	\item 
	\href{https://github.com/jlgarridol/TFG-SmartBeds/issues/83}{Escribir memoria - Herramientas}
	\item 
	\href{https://github.com/jlgarridol/TFG-SmartBeds/issues/84}{Error en conexiones por request: MySQL Connection Not Available}~[BUG]
\end{enumerate}

\subsection{\textit{Spint} 18 - 17/05/2019 - 30/05/2019}
Este \textit{sprint}, que duró dos semanas, se centró en hacer algunas mejoras sobre la aplicación así como documentar algunas partes de la memoria.

\begin{enumerate}\addtocounter{enumi}{84}
	\item 
	\href{https://github.com/jlgarridol/TFG-SmartBeds/issues/85}{Guardar probabilidades del imblearn}
	\item 
	\href{https://github.com/jlgarridol/TFG-SmartBeds/issues/86}{Censurar opciones de administración de camas}
	\item 
	\href{https://github.com/jlgarridol/TFG-SmartBeds/issues/87}{Sistema de alertas y control en frontend}
	\item 
	\href{https://github.com/jlgarridol/TFG-SmartBeds/issues/88}{Simular todas las camas con los mismos hilos}
	\item 
	\href{https://github.com/jlgarridol/TFG-SmartBeds/issues/89}{Crear un clasificador aleatorio}
	\item 
	\href{https://github.com/jlgarridol/TFG-SmartBeds/issues/90}{Escribir el abstract de la memoria}
	\item 
	\href{https://github.com/jlgarridol/TFG-SmartBeds/issues/91}{Documentar proceso de instalación}
	\item 
	\href{https://github.com/jlgarridol/TFG-SmartBeds/issues/92}{Concluir ventanas restantes}
\end{enumerate}

\subsection{\textit{Spint} 19 - 31/05/2019 - 13/06/2019}
Este \textit{sprint}, que duró dos semanas, debido a los exámenes finales que ocurrieron en este periodo, se centró en continuar la documentación así como mejorar la interfaz de la aplicación web.

\begin{enumerate}\addtocounter{enumi}{92}
	\item 
	\href{https://github.com/jlgarridol/TFG-SmartBeds/issues/93}{Continuar la documentación}
	\item 
	\href{https://github.com/jlgarridol/TFG-SmartBeds/issues/94}{Reparar varios bugs}
	\item 
	\href{https://github.com/jlgarridol/TFG-SmartBeds/issues/95}{Diseñar los test de selenium}
	\item 
	\href{https://github.com/jlgarridol/TFG-SmartBeds/issues/96}{Crear pantalla de carga}
	\item 
	\href{https://github.com/jlgarridol/TFG-SmartBeds/issues/97}{Migrar la sesión y la gestión de usuarios a flask-session y flask-login}
\end{enumerate}

\subsection{\textit{Spint} 20 - 14/06/2019 - 20/06/2019}
Este penúltimo \textit{sprint} se centró en terminar todos los puntos de la documentación.

\begin{enumerate}\addtocounter{enumi}{97}
	\item 
	\href{https://github.com/jlgarridol/TFG-SmartBeds/issues/98}{Corregir anexo y memoria}
	\item 
	\href{https://github.com/jlgarridol/TFG-SmartBeds/issues/99}{Escribir objetivos}
	\item 
	\href{https://github.com/jlgarridol/TFG-SmartBeds/issues/100}{Escribir conceptos teóricos}
	\item 
	\href{https://github.com/jlgarridol/TFG-SmartBeds/issues/101}{Escribir estudio de viabilidad}
	\item 
	\href{https://github.com/jlgarridol/TFG-SmartBeds/issues/102}{Escribir manual del usuario}
	\item 
	\href{https://github.com/jlgarridol/TFG-SmartBeds/issues/103}{Escribir trabajos relacionados}
	\item 
	\href{https://github.com/jlgarridol/TFG-SmartBeds/issues/104}{Escribir conclusiones y líneas futuras}
	\item 
	\href{https://github.com/jlgarridol/TFG-SmartBeds/issues/105}{Escribir aspectos relevantes}
\end{enumerate}

\subsection{\textit{Spint} 21 - 21/06/2019 - 27/06/2019}
En este último \textit{sprint} se realizaron las últimas partes del proyecto como las correcciones de algunos bugs y de la documentación

\begin{enumerate}\addtocounter{enumi}{105}
	\item 
	\href{https://github.com/jlgarridol/TFG-SmartBeds/issues/106}{Documentar resultados de las pruebas de usabilidad}
	\item 
	\href{https://github.com/jlgarridol/TFG-SmartBeds/issues/107}{Arreglar últimos detalles de la memoria y anexos}
	\item 
	\href{https://github.com/jlgarridol/TFG-SmartBeds/issues/108}{Reparar \textit{bug} concurrencia}
	\item 
	\href{https://github.com/jlgarridol/TFG-SmartBeds/issues/109}{Añadir el cuaderno de investigación}
	\href{https://github.com/jlgarridol/TFG-SmartBeds/issues/110}{Hacer cambios de la memoria y anexos según los consejos de los tutores}
	\href{https://github.com/jlgarridol/TFG-SmartBeds/issues/111}{Mejorar aspecto README}
\end{enumerate}

\section{Estudio de viabilidad}

\subsection{Viabilidad económica}

Debido a que este TFG se ha realizado en un proyecto de investigación de varias personas, este apartado será común al del TFG de Alicia Olivares Gil, al ser parte del equipo de investigación.

\subsubsection{Costes}

Los costes de personal se desglosarán en las siguientes categorías: costes de personal para el salario de dos personas, el coste hardware de las herramientas utilizadas, coste software del software utilizada. También se incluirán gastos aproximados del material que se ha dispuesto que ya se poseía.

El coste de personal se puede ver en la tabla~\ref{tab:costes_personal}
 
\begin{table}\centering
	\begin{tabular}[]{@{}l r@{}}
		\toprule
		\textbf{Concepto} & \textbf{Coste(\euro{})} \\
		\midrule
		Salario mensual neto & 1.225,7~\cite{salariales} \\
		Retención IRPF (15\%) & 216,3 \\
		Seguridad Social (28,3\%) & 569,16 \\
		Salario mensual bruto & 2.011,16 \\\hubu
		\textbf{Total 7 meses y dos empleados} &  28.156,24 \\
		\bottomrule
	\end{tabular}
	\caption{Costes de personal.}
	\label{tab:costes_personal}
\end{table}

Para el desarrollo de este proyecto no se ha adquirido ningún \textit{hardware} nuevo, por lo que únicamente se incluirán los costes del material con el que ya se contaba asumiendo una amortización en 5 años, y calculando solo el coste de amortización correspondiente a la duración del proyecto (7 meses). Se puede ver en la tabla~\ref{tab:costes_hardware}

\begin{table}
	\centering
	\begin{tabular}[]{@{}l c r@{}}
		\toprule
		\textbf{Concepto} & \textbf{Coste(\euro{})} & \textbf{Coste amortizado(\euro{})} \\
		Dispositivo móvil & 150 & 17,5 \\
		Ordenador portátil (x2) & 800 & 93,33 \\
		\textit{MainFrame} & 3.000 & 350 \\ 
		GPU (x3) & 4.500 & 525 \\\hubu
		\textbf{Total} & 8.450 & 985,83
	\end{tabular}
	\caption{Costes de \textit{hardware}.}
	\label{tab:costes_hardware}.
\end{table}

No ha habido costes software ya que todas las herramientas utilizadas han sido gratuitas o de código abierto.

Teniendo en cuenta los costes de personal y de \textit{hardware}, el coste económico total se puede ver en la tabla~\ref{tab:coste_total}

\begin{table}
	\centering
	\begin{tabular}[]{@{}l r@{}}
		\toprule
		\textbf{Concepto} & \textbf{Coste(\euro{})} \\
		Coste de personal & 28.156,24 \\ 
		Coste del \textit{hardware} & 985,83 \\\hubu
		\textbf{Total} & \textbf{29.142,07} \\		
	\end{tabular}
	\caption{Coste total.}
	\label{tab:coste_total}
\end{table}


\subsection{Viabilidad legal}

En este proyecto se ha realizado con la ayuda de software de terceros con licencias propias que influyen sobre la viabilidad legal del proyecto.

El software utilizado se puede encontrar en el apartado \textit{`Técnicas y herramientas'} de la memoria. Las licencias que utilizan son:

\begin{itemize}
	\item \textbf{MIT}: Esta licencia permite el uso comercial del producto, la modificación del mismo, la libre distribución y el uso privado. No tiene garantías ni responsabilidad. La única condición es hacer referencia a ella. Como no obliga a mantener la licencia ni afecta a la distribución del software que use la licencia final del producto, esta puede ser cualquiera.
	
	\item \textbf{AGPL}: Esta licencia permite el uso comercial, la libre distribución y modificación. Así también permite el uso de patentes y el uso privado. Entre sus condiciones están que el software debe distribuirse con su código, hacer referencia a la licencia, se deben documentar los cambios en el código y utilizar la misma licencia o una compatible (como GPLv3). Tiene en sus limitaciones la garantía y la responsabilidad.
	
	\item \textbf{GPLv2}: Tiene las mismas características que AGPL, cambia la cláusula de patentes sin prohibirlas, no especifica nada de ellas. Tampoco obliga a distribuir el código vía aplicación web.	
	
	\item \textbf{BSD Simplified}: Tiene características semejantes a MIT en el contexto en el que la usamos.
	
	\item \textbf{BSD 3-Clause}: Tiene características semejantes a MIT en el contexto en el que la usamos.
\end{itemize}

Entre todas estas licencias la más restrictiva es AGPL que será la que se use en esta aplicación y todas son compatibles entre si.

