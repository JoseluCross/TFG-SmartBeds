\apendice{Plan de Proyecto Software}

\section{Introducción}

En este apéndice se expondrán los distintos \textit{sprints} que se han realizado y un estudio de la viabilidad del proyecto.

\section{Planificación temporal}

El proyecto se desarrolló siguiendo una metodología ágil basada en \textit{Scrum}. Se dividió el progreso en \textit{Sprints}, cada uno con una serie de tareas y su estimación en esfuerzo. Esta estimación o peso, se ha evaluado según la dificultad que se preveía tener, no como una medida horaria, esto es debido que algunas tareas simples tardan más en desarrollarse por la necesidad de la ejecución automática mientras que otras que requirieron mucho más trabajo se ejecutaron mucho antes. Algunas tareas además se consideran épicas, estas son resúmenes de un proceso completo que abarca varias tareas y duran generalmente más de un \textit{sprint}. 

A continuación se mostrarán las listas de las tareas realizadas con los enlaces a las \textit{issues} del repositorio \textit{\href{https://github.com/jlgarridol/TFG-SmartBeds}{GitHub}}. Los \textit{Sprints} fueron:
\subsection{\textit{Sprint} 1 - 10/11/18 hasta 20/11/18}
En este primer \textit{Sprint} solamente hubo dos tareas:
\begin{enumerate}
	\item \href{https://github.com/joselucross/TFG-SmartBeds/issues/1}{Hacer exploración bibliográfica}
	\item \href{https://github.com/joselucross/TFG-SmartBeds/issues/2}{Configurar repositorio de Git}
\end{enumerate}
\subsection{\textit{Sprint} 2 - 21/11/18 hasta 05/12/18}
Se desarrollaron 4 tareas, todas de investigación, por lo que no hay \textit{commits} asociados, para esto las tareas fueron:

\begin{enumerate}\addtocounter{enumi}{2}
	\item \href{https://github.com/joselucross/TFG-SmartBeds/issues/3}{Lectura de <<Automated Epileptic Seizure Detection Methods: A Review Study>>}
	\item \href{https://github.com/joselucross/TFG-SmartBeds/issues/4}{Exploración Bibliográfica - Orientado a caídas}
	\item \href{https://github.com/joselucross/TFG-SmartBeds/issues/5}{ Lectura del primer tema de "Minería de Datos"}
	\item \href{https://github.com/joselucross/TFG-SmartBeds/issues/6}{ Configurar VPN en Archlinux}
\end{enumerate}

\subsection{\textit{Sprint} 3 - 06/12/18 hasta 21/12/18}
En este \textit{sprint} se comenzó la documentación migrando las plantillas al repositorio y continuó la investigación a un área más técnica, las tareas fueron:

\begin{enumerate}\addtocounter{enumi}{6}
	\item \href{https://github.com/joselucross/TFG-SmartBeds/issues/7}{Iniciar documentación}
	\item \href{https://github.com/joselucross/TFG-SmartBeds/issues/8}{Tabla de extracción de características}
	\item \href{https://github.com/joselucross/TFG-SmartBeds/issues/9}{ Búsqueda de librerías con funciones sofisticadas}
	
\end{enumerate}

\subsection{\textit{Sprint} 4 - 22/12/18 hasta 08/01/2019}
Este \textit{sprint} fue del aprendizaje de técnicas de minería de datos aplicada. Las tareas realizadas fueron:

\begin{enumerate}\addtocounter{enumi}{9}
	\item \href{https://github.com/joselucross/TFG-SmartBeds/issues/10}{ Instalación de entorno Python}
	\item \href{https://github.com/joselucross/TFG-SmartBeds/issues/11}{ Graficar datos mediante PCA}
	\item \href{https://github.com/joselucross/TFG-SmartBeds/issues/12}{ Aprender el uso de librerías}
\end{enumerate}

\subsection{\textit{Sprint} 5 - 09/01/2019 hasta 17/01/2019}
En este \textit{sprint} el objetivo fue probar distintas lineas de investigación para ver las características intrínsecas a los datos.

\begin{enumerate}\addtocounter{enumi}{12}
	\item \href{https://github.com/joselucross/TFG-SmartBeds/issues/13}{ Configurar acceso a gamma}
	\item \href{https://github.com/joselucross/TFG-SmartBeds/issues/14}{ Leer apuntes de Minería de Datos}
	\item \href{https://github.com/joselucross/TFG-SmartBeds/issues/15}{ Filtrado y suavizado de datos}
	\item \href{https://github.com/joselucross/TFG-SmartBeds/issues/16}{ Probar otras formas de proyección}
\end{enumerate}	

\subsection{\textit{Sprint} 6 - 18/01/2019 hasta 24/01/2019}
Tras descartar algunas proyecciones se exploraron las más prometedores y se probaron nuevos filtros y detección de anomalías. Las tareas fueron por consiguiente:

\begin{enumerate}\addtocounter{enumi}{16}
	\item \href{https://github.com/joselucross/TFG-SmartBeds/issues/17}{ Mejor preprocesamiento}
	\item \href{https://github.com/joselucross/TFG-SmartBeds/issues/18}{ Dibujado alrededor de las crisis}
	\item \href{https://github.com/joselucross/TFG-SmartBeds/issues/19}{ Probar formas de filtrado}
	\item \href{https://github.com/joselucross/TFG-SmartBeds/issues/20}{ Estudiar puntos clave de las proyecciones}
	\item \href{https://github.com/joselucross/TFG-SmartBeds/issues/21}{ Probar detección de anomalías por one-class}
\end{enumerate}

\subsection{\textit{Sprint} 7 - 25/01/2019 hasta 07/02/2019}
Este \textit{sprint} tuvo una duración mayor debido a que durante el periodo señalado se realizó un curso en la universidad donde se estudiaron las series temporales. Por este motivo, el servidor donde se han estado realizando las ejecuciones no estaba disponible retrasando la ejecución de los experimentos.

Las tareas se centraron en comprobar las proyecciones más interesantes con datos estadísticos además de profundizar en \textit{One-Class} o detección de anomalías. Las tareas realizadas fueron:
\begin{enumerate}\addtocounter{enumi}{21}
	\item \href{https://github.com/joselucross/TFG-SmartBeds/issues/22}{Proyección LTSA con valores estadísticos}~\cite{zhang2004principal}
	\item 
	\href{https://github.com/joselucross/TFG-SmartBeds/issues/23}{Documentar Sprints del 1 al 6}
	\item
	\href{https://github.com/joselucross/TFG-SmartBeds/issues/24}{Documentar investigación con Alicia Olivares}
	\item
	\href{https://github.com/joselucross/TFG-SmartBeds/issues/25}{One Class - Mejorar la investigación de este apartado}
	\item
	\href{https://github.com/joselucross/TFG-SmartBeds/issues/26}{Transformers}
	\item
	\href{https://github.com/joselucross/TFG-SmartBeds/issues/27}{Trasladar códigos a transformers}
\end{enumerate}

\subsection{\textit{Sprint} 8 - 8/02/2019 hasta 14/2/2019}
Este \textit{sprint} fue dedicado a seguir desarrollando el estudio de \textit{One Class} cuyos resultados se pueden ver en el apéndice \textit{Cuaderno de Investigación}. Las tareas fueron:

\begin{enumerate}\addtocounter{enumi}{27}
	\item \href{https://github.com/joselucross/TFG-SmartBeds/issues/28}{Particionar los datos}
	\item \href{https://github.com/joselucross/TFG-SmartBeds/issues/29}{Estudiar one-class\footnote{Tarea épica, siendo estas las tareas que engloban varias tareas más simples}}
	\item \href{https://github.com/joselucross/TFG-SmartBeds/issues/30}{One-Class con crisis}
	\item \href{https://github.com/joselucross/TFG-SmartBeds/issues/31}{One-Class sin crisis}
	\item \href{https://github.com/joselucross/TFG-SmartBeds/issues/32}{Preprocesado básico para \textit{One Class}}
	\begin{enumerate}
		\item Bruto
		\item Filtrado \textit{Butter} de 3 y 0.05
		\item Filtrado \textit{SavGol} de tamaño 15
		\item Concatenación de estadísticos en ventana 25 sobre bruto
		\item Concatenación de estadísticos con ventana 25 sobre \textit{Butter} de 3 y 0.05
		\item Concatenación de estadísticos con ventana 25 sobre \textit{SavGol} de tamaño 15
	\end{enumerate}
	\item \href{https://github.com/joselucross/TFG-SmartBeds/issues/33}{Testear clasificadores con otras crisis}
\end{enumerate}
Las tareas de la 30 a la 33 se incluyeron en la tarea épica 29 aunque esta tuvo tareas del siguiente \textit{sprint}

\subsection{\textit{Sprint} 9 - 15/2/2019 a 21/2/2019}
Este \textit{sprint} trató de completar la linea de investigación de \textit{One Class} abierta en la tarea épica 29 y documentar los resultados de \textit{sprints} anteriores. 

\begin{enumerate}\addtocounter{enumi}{33}
	\item
	\href{https://github.com/joselucross/TFG-SmartBeds/issues/34}{Documentar PCA y Proyecciones de 2-Variedad}
	\item
	\href{https://github.com/joselucross/TFG-SmartBeds/issues/35}{Entrenamiento \textit{One Class} con dos crisis}
	\item
	\href{https://github.com/joselucross/TFG-SmartBeds/issues/36}{Testeo \textit{One Class} con tercera crisis}
	\item
	\href{https://github.com/joselucross/TFG-SmartBeds/issues/37}{Calcular área bajo la curva con entrenamiento de una clase}
	\item
	\href{https://github.com/joselucross/TFG-SmartBeds/issues/38}{Visualización de constantes vitales}
\end{enumerate}
Son las tareas de la 35 a la 37 las que forman parte de la tarea épica 29.
\subsection{\textit{Sprint} 10 - 22/3/2019 a 28/3/2019}
Este \textit{sprint} estuvo centrado en la lectura y aprendizaje de nuevos métodos

\begin{enumerate}\addtocounter{enumi}{38}
	\item
	\href{https://github.com/joselucross/TFG-SmartBeds/issues/39}{Investigar sobre desequilibrados}
	\item
	\href{https://github.com/joselucross/TFG-SmartBeds/issues/40}{Aprender Weka}
	\item
	\href{https://github.com/joselucross/TFG-SmartBeds/issues/41}{Documentar \textit{One Class}}
\end{enumerate}
\subsection{\textit{Sprint} 11 - 01/03/2019 a 21/03/2019}
Este \textit{sprint} tuvo una duración mayor debido a la diversidad de experimentos y la presencia de varios días no lectivos. Se centró en el lanzamiento de experimentos con \textit{ensembles}.
\begin{enumerate}\addtocounter{enumi}{41}
	\item
	\href{https://github.com/joselucross/TFG-SmartBeds/issues/42}{Creación de test de transformers}
	\item
	\href{https://github.com/joselucross/TFG-SmartBeds/issues/43}{Filtrado SMOTE}
	\item
	\href{https://github.com/joselucross/TFG-SmartBeds/issues/43}{Pruebas con ensembles ya implementados}
	\item
	\href{https://github.com/joselucross/TFG-SmartBeds/issues/45}{Prueba de ensembles para desequilibrados}
\end{enumerate}

\subsection{\textit{Sprint} 12 - 22/03/2019 - 28/03/2019}
En esta ocasión comenzamos a reducir el esfuerzo sobre la investigación sobre un incremento en el diseño de la aplicación.

\begin{enumerate}\addtocounter{enumi}{45}
	\item
	\href{https://github.com/joselucross/TFG-SmartBeds/issues/46}{Exploración de herramientas}
	\item
	\href{https://github.com/joselucross/TFG-SmartBeds/issues/47}{Documentar resultados anteriores}
	\item
	\href{https://github.com/joselucross/TFG-SmartBeds/issues/48}{Diseño del servidor}
	\item
	\href{https://github.com/joselucross/TFG-SmartBeds/issues/49}{Ejecutar experimentos Weka restantes}
\end{enumerate}

\subsection{\textit{Sprint} 13 - 29/03/2019 - 08/04/2019}
Este \textit{sprint} se centró en la definición de requisitos y de realizar experimentos con los resultados de Alicia Olivares. Sin embargo, estos experimentos no pudieron ser realizados al descubrir un fallo en la investigación previa, por lo que no se realizaron quedando pendientes ante nuevos datos. Por tanto, las tareas que realmente fueron realizadas fueron:

\begin{enumerate}\addtocounter{enumi}{49}
	\item
	\href{https://github.com/joselucross/TFG-SmartBeds/issues/50}{Simulador de la cama}
	\addtocounter{enumi}{1}
	\item
	\href{https://github.com/joselucross/TFG-SmartBeds/issues/52}{Diseño de casos de uso}
	\item
	\href{https://github.com/joselucross/TFG-SmartBeds/issues/53}{Diseño de los datos para la API}
	\item
	\href{https://github.com/joselucross/TFG-SmartBeds/issues/54}{Diseño de la base de datos}
	\addtocounter{enumi}{2}
	\item
	\href{https://github.com/joselucross/TFG-SmartBeds/issues/56}{Diseño de la API}
	\item
	\href{https://github.com/joselucross/TFG-SmartBeds/issues/57}{Requisitos funcionales}
\end{enumerate} 

\subsection{\textit{Spint} 14 - 09/04/2019 - 11/04/2019 }
Este \textit{sprint} se concluyeron la creación de los experimentos que quedaron pendientes en el \textit{sprint} anterior como seguir desarrollando componentes de la aplicación. 

Sin embargo, muchas partes de este \textit{sprint} fueron quedando en \textit{icebox} tras algunas complicaciones en los experimentos y la nueva linea de investigación con \textit{PRC} que se puede ver en el \textit{Cuaderno de investigación} adjuntado. 

\begin{enumerate}\addtocounter{enumi}{50}
	\item
	\href{https://github.com/jlgarridol/TFG-SmartBeds/issues/51}{Experimentos Weka con los datos de series temporales}
	\addtocounter{enumi}{3}
	\item
	\href{https://github.com/jlgarridol/TFG-SmartBeds/issues/55}{Implementación de la base de datos}
	\addtocounter{enumi}{2}
	\href{https://github.com/jlgarridol/TFG-SmartBeds/issues/58}{Creación de experimentos}
	\item
	\href{https://github.com/jlgarridol/TFG-SmartBeds/issues/59}{Lanzamiento de experimentos}
	\item
	\href{https://github.com/jlgarridol/TFG-SmartBeds/issues/60}{Documentación de resultados}
	\item
	\href{https://github.com/jlgarridol/TFG-SmartBeds/issues/61}{Experimentos Desequilibrador\footnote{Tarea épica}}
	\item
	\href{https://github.com/jlgarridol/TFG-SmartBeds/issues/62}{Ultimar detalles del cuaderno de trabajo}
	\item
	\href{https://github.com/jlgarridol/TFG-SmartBeds/issues/63}{Arreglar documentación de casos de uso}
	\item
	\href{https://github.com/jlgarridol/TFG-SmartBeds/issues/64}{Completar diseño de bases de datos}
	\item
	\href{https://github.com/jlgarridol/TFG-SmartBeds/issues/65}{Trabajar en formato de latex}
	\item 
	\href{https://github.com/jlgarridol/TFG-SmartBeds/issues/66}{Crear interfaz de la api}
	\item 
	\href{https://github.com/jlgarridol/TFG-SmartBeds/issues/67}{Creación y documentación de prototipos de interfaz}
\end{enumerate}

\subsection{\textit{Spint} 15 - 12/04/2019 - 2/05/2019 }
En este \textit{sprint} se comenzó la implementación de la API así como la realización de test automáticos de la misma. También se comenzó a preparar el servidor para que fuese accesible desde Internet.

\begin{enumerate}\addtocounter{enumi}{67}
	\item 
	\href{https://github.com/jlgarridol/TFG-SmartBeds/issues/68}{Programación de la API}
	\item 
	\href{https://github.com/jlgarridol/TFG-SmartBeds/issues/69}{Test de API}
	\item 
	\href{https://github.com/jlgarridol/TFG-SmartBeds/issues/70}{Acceso web a la API}
	\item 
	\href{https://github.com/jlgarridol/TFG-SmartBeds/issues/71}{Documentar Sprints anteriores}
	\item 
	\href{https://github.com/jlgarridol/TFG-SmartBeds/issues/72}{Configurar servidor}
	\item 
	\href{https://github.com/jlgarridol/TFG-SmartBeds/issues/73}{Entorno Flask}
\end{enumerate}

\subsection{\textit{Spint} 16 - 3/05/2019 - 09/05/2019}
Este \textit{sprint} consistió en continuar el desarrollo del servidor.

\begin{enumerate}\addtocounter{enumi}{73}
	\item 
	\href{https://github.com/jlgarridol/TFG-SmartBeds/issues/74}{Hilo de procesamiento}
	\item 
	\href{https://github.com/jlgarridol/TFG-SmartBeds/issues/75}{Distribución de paquetes por SocketIO}
	\item 
	\href{https://github.com/jlgarridol/TFG-SmartBeds/issues/76}{Uso del clasificador de Alicia Olivares}
	\item 
	\href{https://github.com/jlgarridol/TFG-SmartBeds/issues/77}{Actualización de Chart.js es muy lenta}
	\item 
	\href{https://github.com/jlgarridol/TFG-SmartBeds/issues/78}{Lanzar proyecto en el servidor}
\end{enumerate}

\subsection{\textit{Spint} 17 - 10/05/2019 - 16/05/2019}
Este \textit{sprint} se centró en el desarrollo de la vista así como reparar algunos \textit{bugs} que fueron apareciendo. También se intento utilizar \textit{Heroku} pero al conseguir eliminar las limitaciones del servidor se desechó la idea.

\begin{enumerate}\addtocounter{enumi}{79}
	\item 
	\href{https://github.com/jlgarridol/TFG-SmartBeds/issues/80}{Conexiones SQL por request}
	\item 
	\href{https://github.com/jlgarridol/TFG-SmartBeds/issues/81}{Hacer todas las ventanas}
	\item 
	\href{https://github.com/jlgarridol/TFG-SmartBeds/issues/82}{Crear cama simulada de los datos alrededor de las crisis}
	\item 
	\href{https://github.com/jlgarridol/TFG-SmartBeds/issues/83}{Escribir memoria - Herramientas}
	\item 
	\href{https://github.com/jlgarridol/TFG-SmartBeds/issues/84}{Error en conexiones por request: MySQL Connection Not Available}
\end{enumerate}

\subsection{\textit{Spint} 18 - 17/05/2019 - 30/05/2019}
Este \textit{sprint}, que duró dos semanas, se centró en hacer algunas mejoras sobre la aplicación así como documentar algunas partes de la memoria.

\begin{enumerate}\addtocounter{enumi}{84}
	\item 
	\href{https://github.com/jlgarridol/TFG-SmartBeds/issues/85}{Guardar probabilidades del imblearn}
	\item 
	\href{https://github.com/jlgarridol/TFG-SmartBeds/issues/86}{Censurar opciones de administración de camas}
	\item 
	\href{https://github.com/jlgarridol/TFG-SmartBeds/issues/87}{Sistema de alertas y control en frontend}
	\item 
	\href{https://github.com/jlgarridol/TFG-SmartBeds/issues/88}{Simulación falseada de las camas}
	\item 
	\href{https://github.com/jlgarridol/TFG-SmartBeds/issues/89}{Clasificador aleatorio}
	\item 
	\href{https://github.com/jlgarridol/TFG-SmartBeds/issues/90}{Escribir el abstract de la memoria}
	\item 
	\href{https://github.com/jlgarridol/TFG-SmartBeds/issues/91}{Documentar proceso de instalación}
	\item 
	\href{https://github.com/jlgarridol/TFG-SmartBeds/issues/92}{Concluir ventanas restantes}
\end{enumerate}

\section{Estudio de viabilidad}

\subsection{Viabilidad económica}

\subsection{Viabilidad legal}


